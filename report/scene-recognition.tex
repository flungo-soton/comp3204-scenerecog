%%% Preamble
\documentclass[paper=a4, fontsize=11pt]{article}
\begin{document}
\title{Scene Recognition}
\author{Jianzu Guan (jg17g13) and Fabrizio Lungo (fl4g12)}
\maketitle

\section{Overview}

In this project we will be implementing algorithms in order to classify images based on the scene that they show. We have bee provided with 1500 labelled training images from 15 classes (100 images per class). If we were to randomly guess and assign classes to the images, we would expect to have an accuracy of approximately $6.67\%$.

The implementations will be implemented in Java and utilising the OpenIMAJ library as best as possible.

\section{Framework}

In order to assist with the running and evaluation of the classification implementations and reduce code duplication a framework has been implemented.

To get the data (either from the remote URL or locally, if available) a \texttt{DatasetUtil} is provided which is able to give the labelled training set and unlabelled test sets.

% TODO: Actually produce classifier, need to update code and class names to reflect this.
All of the classifiers eventually produce an \texttt{Annotator} which assigns a class to each image. This can be used to abstract the evaluation and usage of the implementation. Since different assigners have different methods to be trained, this is abstracted through an \texttt{AnnotatorWrapper} which provides a train method taking the unlabelled dataset. The wrapper can handle any implementation specific features and provide an interface which is generic for all tests.

In order to write the outputs of classifications to the respective output file in the format required, the \texttt{GroupWriter} class takes a \texttt{GroupedDataset} and writes the classifications in the required format.

An abstracted base class \texttt{Classification} has been created which is able to train, test and evaluate an \texttt{Annotator}. An abstract \texttt{getAnnotatorWrapper} method is declared which should be fulfilled by the specific classifications. It uses the \texttt{AnnotatorWrapper} returned to interface with the specific annotator to evaluate, train and test.

Evaluation uses the \texttt{ClassificationEvaluator} provided by OpenIMAJ in order to validate the training data. The data is split using a \texttt{StratifiedGroupedRandomisedPercentageHoldOut} which splits the data ensuring that the training and validation ratio is consistent for each class.

The train and test will use all of the training data and then test the unlabelled test data using a \texttt{DatasetClassifier} to classify the data into a \texttt{GroupedDataset} and the appropriate \texttt{GroupWriter} to write the output the results to file. In order to maintain file names, the images from the testing data are wrapped with \texttt{IdentifyableObject}.

All classes have been implemented using Java Generics where appropriate in order to make them as reusable and robust as possible.
\end{document}